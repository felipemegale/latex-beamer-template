%%%%%%%%%%%%%%%%%%%%%%%%%%%%%%%%%%%%%%%%%
% Beamer Presentation
% LaTeX Template
% Version 1.0 (10/11/12)
%
% This template has been downloaded from:
% http://www.LaTeXTemplates.com
%
% License:
% CC BY-NC-SA 3.0 (http://creativecommons.org/licenses/by-nc-sa/3.0/)
%
%%%%%%%%%%%%%%%%%%%%%%%%%%%%%%%%%%%%%%%%%

%----------------------------------------------------------------------------------------
%	PACKAGES AND THEMES
%----------------------------------------------------------------------------------------

\documentclass{beamer}

\mode<presentation> {
    \usetheme{Pittsburgh}
    \usecolortheme{seahorse}
    %\setbeamertemplate{footline} % To remove the footer line in all slides uncomment this line
    \setbeamertemplate{footline}[page number] % To replace the footer line in all slides with a simple slide count uncomment this line
    \setbeamertemplate{navigation symbols}{} % To remove the navigation symbols from the bottom of all slides uncomment this line
}

\usepackage{amssymb}
\usepackage[font=small,labelfont=bf]{caption}
\usepackage{graphicx} % Allows including images
\usepackage{booktabs} % Allows the use of \toprule, \midrule and \bottomrule in tables
%\usepackage {tikz}
\usepackage{tkz-graph}
\GraphInit[vstyle = Shade]
\tikzset{
  LabelStyle/.style = { rectangle, rounded corners, draw,
                        minimum width = 2em, fill = yellow!50,
                        text = red, font = \bfseries },
  VertexStyle/.append style = { inner sep=5pt,
                                font = \normalsize\bfseries},
  EdgeStyle/.append style = {->, bend left} }
\usetikzlibrary {positioning}
%\usepackage {xcolor}
\definecolor {processblue}{cmyk}{0.96,0,0,0}

%----------------------------------------------------------------------------------------
%	TITLE PAGE
%----------------------------------------------------------------------------------------

\title[Presentation title]{Presentation title} % The short title appears at the bottom of every slide, the full title is only on the title page

\author{Your name} 

\institute[Institution for other slides]
{
    Institution for title slide \\
    \medskip
}

%\date{\today} % Date, can be changed to a custom date
\date{September 29, 2021}

%\logo{
%    \includegraphics[height=1cm]{imgs/logo.png} % use here your University logo
%}

\begin{document}

\begin{frame}
    \titlepage % Print the title page as the first slide
\end{frame}

\begin{frame}
    \frametitle{Overview} % Table of contents slide, comment this block out to remove it
    \tableofcontents % Throughout your presentation, if you choose to use \section{} and \subsection{} commands, these will automatically be printed on this slide as an overview of your presentation
\end{frame}

%----------------------------------------------------------------------------------------
%	PRESENTATION SLIDES
%----------------------------------------------------------------------------------------

%------------------------------------------------

\section{Slide example 1}
\begin{frame}{Slide example 1}
    \begin{itemize}
        \item Level 1 item
        \begin{itemize}
            \item Level 2 item
        \end{itemize}
    \end{itemize}
\end{frame}

%------------------------------------------------

\section{Math example}
\begin{frame}{Use \$something\$ to get math symbols}
    \begin{itemize}
        \item Like so: $\phi$
    \end{itemize}
\end{frame}

%------------------------------------------------

\section{Image example}
\begin{frame}{Image example}
    \begin{figure}
        \centering
        %\includegraphics[width=.9\textwidth]{imgs/fig1_a.jpg} % add an img that exists in imgs folder
        \captionof{figure}{Figure caption}
        \label{fig:1}
    \end{figure}
\end{frame}

%------------------------------------------------

\section{References}
\begin{frame}
    \frametitle{References}
    \footnotesize{
        \begin{thebibliography}{99} % Beamer does not support BibTeX so references must be inserted manually as below
            \bibitem[Author et al., YEAR]{p1} Authors full names and year
            \newblock Work title
            \newblock \emph{Where work was published} What Volume or Pages, publication date.
        \end{thebibliography}
    }
\end{frame}

%------------------------------------------------
% Thank you slide
\begin{frame}
\Huge{\centerline{Thanks!}}
\end{frame}

%------------------------------------------------

\end{document}

%------------------------------------------------

%\begin{frame}
%\frametitle{Multiple Columns}
%\begin{columns}[c] % The "c" option specifies centered vertical alignment while the "t" option is used for top vertical alignment

%\column{.45\textwidth} % Left column and width
%\textbf{Heading}
%\begin{enumerate}
%\item Statement
%\item Explanation
%\item Example
%\end{enumerate}

%\column{.5\textwidth} % Right column and width
%Lorem ipsum dolor sit amet, consectetur adipiscing elit. Integer lectus nisl, ultricies in feugiat rutrum, porttitor sit amet augue. Aliquam ut tortor mauris. Sed volutpat ante purus, quis accumsan dolor.

%\end{columns}
%\end{frame}

%------------------------------------------------
%\section{Second Section}
%------------------------------------------------

%\begin{frame}
%\frametitle{Table}
%\begin{table}
%\begin{tabular}{l l l}
%\toprule
%\textbf{Treatments} & \textbf{Response 1} & \textbf{Response 2}\\
%\midrule
%Treatment 1 & 0.0003262 & 0.562 \\
%Treatment 2 & 0.0015681 & 0.910 \\
%Treatment 3 & 0.0009271 & 0.296 \\
%\bottomrule
%\end{tabular}
%\caption{Table caption}
%\end{table}
%\end{frame}

%------------------------------------------------

%----------------------------------------------------------------------------------------
